\documentclass[a4paper,oneside,11pt]{article}
\usepackage{amssymb}
\usepackage{overcite}
\usepackage{amsmath}
\usepackage{enumerate}
\usepackage{graphicx}
\usepackage[nottoc]{tocbibind}
\usepackage{setspace}
\usepackage{rotating}
%\doublespacing
\parindent=0cm
%\parskip=.5cm
\numberwithin{equation}{section}

\newcommand{\rff}[1]{{Eq.~\ref{#1}}}

\newcommand{\half}{\ensuremath{\frac{1}{2}}}
\newcommand{\Rop}{\ensuremath{\mathbf{R}}}
\newcommand{\wfn}[2]{\ensuremath{\psi_{#1}{(#2)}}}
\newcommand{\VEC}[1]{\ensuremath{\mathbf{#1}}}
\newcommand{\pvec}{\ensuremath{\mathbf{p}}}
\newcommand{\rvec}{\ensuremath{\mathbf{r}}}
\newcommand{\kvec}{\ensuremath{\mathbf{k}}}
\newcommand{\Lvec}{\ensuremath{\mathbf{L}}}
\newcommand{\omegavec}{\mathbf{\omega}}
\newcommand{\drdr}{d\mathbf{r}_1d\mathbf{r}_2}
\newcommand{\latvec}[1]{\ensuremath{\mathbf{a}_{#1}}}
\newcommand{\Hamil}{\hat{H}}
\newcommand{\hamil}{\hat{h}}
\newcommand{\Tr}{\mathrm{Tr}}
\newcommand{\Det}[1]{\ensuremath{D_{\mathbf{#1}}}}
\newcommand{\w}[1]{\ensuremath{w_{\mathbf{#1}}}}
\newcommand{\Etilde}[1]{\ensuremath{\tilde{E}_{\mathbf{#1}}}}
\newcommand{\vxc}{\ensuremath{\hat{v}_{\text{xc}}}}
\newcommand{\vext}{\ensuremath{\hat{v}_{\text{ext}}}}
\newcommand{\vhar}{\ensuremath{\hat{v}_{\text{har}}}}
\newcommand{\eigv}[1]{\ensuremath{\varepsilon_{#1}}}
\newcommand{\htdop}{\ensuremath{e^{-\beta\Hamil/P}}}
\newcommand{\rhoel}[1]{\ensuremath{\rho_{\mathbf{#1}}}}
\newcommand{\cyc}[1]{\ensuremath{(\mathbf{#1})}}
\def\Braket#1{\mathinner{\left\langle{#1}\right\rangle}}
\def\braket#1{\mathinner{\langle{#1}\rangle}}
\renewcommand{\thefootnote}{\fnsymbol{footnote}}

\def \bfX {{\bf X}}
\def \bfr {{\bf r}}
\def \bfR {{\bf R}}
\def \bfi {{\bf i}}
\def \ket {{\rangle}}
\def \bra {{\langle}}

\begin{document}

\title{Notes on Linear Response Theory for DMET}
\date{}
\maketitle

\section{Linear Response Theory}
To work more on - just LR equations
Form of V and A
Non-interacting theory, CIS and RPA
\section{Background}
DMET specific points
Strongly contracted space
V local
\subsection{Spaces \& Notation}
Notation, and spaces to consider
\section{Implementation}
\subsection{Hessian construction}
Go through each term
Subsubsection on reduction of ai terms to non-interacting limit
\subsection{Overlap}
Overlap diagonal is zero - reasons
How final equations are exactly constructed
Removal of zeroth order wavefunction
\subsection{Transition moments}
...and construction of excitation spectrum
\section{Issues/Tests}
\section{Further work}

\end{document}

%\documentclass[aps,prl,twocolumn,nobibnotes]{revtex4}
\documentclass[aps,showpacs,twocolumn,nobibnotes]{revtex4}
%\documentclass[aps,preprint,showpacs,nobibnotes]{revtex4}
%\documentclass[aps,preprint,nobibnotes]{revtex}
\usepackage{graphics,graphicx,amsfonts,amsmath,amsbsy,amssymb,color}
\usepackage{bm}
%\usepackage{epic}
%\usepackage{mciteplus}
\usepackage{subfigure}
\usepackage{vector}  % Allows "\bvec{}" and "\buvec{}" for "blackboard" style bold vectors in maths
\newcommand{\D}[1] {D_{\bf #1}}
\def \beq {\begin{eqnarray}}
\def \eeq {\end{eqnarray}}
\def \Schrodinger {{Schr\"{o}dinger }}
\def \Di {{D_{\bfi}}}
\def \Dj {{D_{\bfj}}}
\newcommand {\adag}[1] {{a_{#1}^\dagger}}
\def \bfj {{\bf j}}
\def \bfi {{\bf i}}
\def \rone {{\bvec{r}_1}}
\def \rtwo {{\bvec{r}_2}}
\def \mEh {{\textrm{mE}_{\textrm{h}}}}
\def \Eh {{\textrm{E}_{\textrm{h}}}}
\def \nadd {{n_a}}
\newcommand{\braket}[3] {{\langle #1 | #2 | #3 \rangle}}
\newcommand{\brket}[2] {{\langle #1 | #2 \rangle}}
\newcommand{\bra}{\ensuremath{\langle}}
\newcommand{\ket}{\ensuremath{\rangle}}
%\def \ham {{\bf H}}
\def \ham {{\hat{H}}}
%\def \Sz {{\hat{\textrm{S}_{\textrm{z}}}}}
\def \Sz {{\hat{S}_z}}
\newcommand{\rff}[1]{{Eq.~\eqref{#1}}}
\def \Pgen {{P_{\textrm{gen}}}}
\def \Carb {{\textrm{C}_{\textrm{2}}}}
\def \Hij {{H_{\bvec{i}\bvec{j}}}}
\def \Kii {{K_{\bvec{i}\bvec{i}}}}
\def \Kij {{K_{\bvec{i}\bvec{j}}}}

\begin{document}
\title{Spectral functions of extended systems via quantum embedding}
\author{George~H.~Booth}
%\email{ghb24@cam.ac.uk}
\author{Garnet~Kin-Lic~Chan}  
\affiliation{Department of Chemistry, Frick Laboratory, Princeton University, Princeton, New Jersey 08544, USA}

\begin{abstract}
In a previous publication [PRL {\bf 109} 186404 (2012)], the ground-state density matrix embedding theory (DMET) was introduced. With similarities 
to the dynamical mean-field method, a set of local sites are self-consistently correlated, but with an analytically constructable bath describing the
coupling to the rest of the extended system. 
Despite many formal advantages in the analytic embedding, the method was restricted to static, ground-state properties.
Here, we extend the approach to introduce a frequency dependence, and demonstrate accurate spectral functions at a small 
computational cost. In a similar spirit to the frequency independent DMET method, the impurities are coupled via a small set of 
analytically constructed, but now frequency-dependent bath states. In contrast to dynamical mean-field theory, the resultant 
spectral functions are directly obtained on the real-frequency axis, with no bath discretization and allow for straightforward generalization both 
to impurity clusters and arbitrary electron perturbation operators. We demonstrate
the application of this method on the hubbard model, where both the the Kondo resonances and metal-insulator transitions are well reproduced, as well as 
demonstrating optical spectra. This vastly extends the scope and applicability 
of the DMET method in condensed matter problems as a cheap route to correlated local spectral functions of extended systems.
\end{abstract}
\date{\today}
\maketitle

Dynamic correlation functions are directly probed in spectroscopic methods, and correspond to many of the important transport, optical and 
wider electronic structure properties of materials. As such, their accurate theoretical construction is key for materials scientists. 
However, few robust approaches exist for strongly correlated problems. The difficulty is in simultaneously requiring both an accurate 
treatment of the electron correlations beyond mean-field band theory for 
ground states, as well as the excitation spectrum coupled to the infinite bulk
system.

all excited state components at each frequency, as well as 
convergence with respect to the finite-size effects in the solid state. In dynamical mean-field theory (DMFT), the central quantity 
is the greens function of the system, which is self-consistenty optimized 

Difficulty with optical spectra - 
Not constrained by finite size effects which have plagues QMC and ED calculations.

Disadvantages of DMFT:

Size of bath independent of the size of the underlying lattice. Only a function of the impurity cluster size and the perturbation, while still exact in the non-interacting limits.

Coupling only within the bandwidth of the 1 electron bath
Plotted within the bandwidth of the 1 electron 

1 electron response operator is partitioned into the schmidt basis, such that a block acts purely in the schmidt basis of the ground state embedded space, 

Schmidt decompose the action of the non-interacting operator, into that acting on the space of the impurity+bath ground state system, and that acting on the rest of the environment orbitals.

Cost of any frequency point not more than a minute on a single processor core.

DD results using reoptimized ground state.

Within the 1 electron response function bandwidth

Doped Spectral functions in 1D with DMRG: PRL (2004) 256401, 92

ED can only treat small clusters, and therefore the relatively small numbers of poles that are unable to be resolved into the spectral functions of the extended system. Perturbation theory, 
correlation only treated to low order, and will 
lose accuracy as you tend towards to opposite coupling limit to the zeroth order solution (PRL 84 522 1999), and QMC is formulated in imaginary time, and therefore the maximum entropy is needed,
washes out the subtle feature of the spectrum.


%\bibliography{SpectralDMETBib}

\end{document}

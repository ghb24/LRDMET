%\documentclass[aps,prl,twocolumn,nobibnotes]{revtex4}
\documentclass[aps,showpacs,twocolumn,nobibnotes]{revtex4}
%\documentclass[aps,preprint,showpacs,nobibnotes]{revtex4}
%\documentclass[aps,preprint,nobibnotes]{revtex}
\usepackage{graphics,graphicx,amsfonts,amsmath,amsbsy,amssymb,color}
\usepackage{bm}
%\usepackage{epic}
%\usepackage{mciteplus}
\usepackage{subfigure}
\usepackage{vector}  % Allows "\bvec{}" and "\buvec{}" for "blackboard" style bold vectors in maths
\newcommand{\D}[1] {D_{\bf #1}}
\def \beq {\begin{eqnarray}}
\def \eeq {\end{eqnarray}}
\def \Schrodinger {{Schr\"{o}dinger }}
\def \Di {{D_{\bfi}}}
\def \Dj {{D_{\bfj}}}
\newcommand {\adag}[1] {{a_{#1}^\dagger}}
\def \bfj {{\bf j}}
\def \bfi {{\bf i}}
\def \rone {{\bvec{r}_1}}
\def \rtwo {{\bvec{r}_2}}
\def \mEh {{\textrm{mE}_{\textrm{h}}}}
\def \Eh {{\textrm{E}_{\textrm{h}}}}
\def \nadd {{n_a}}
\newcommand{\braket}[3] {{\langle #1 | #2 | #3 \rangle}}
\newcommand{\brket}[2] {{\langle #1 | #2 \rangle}}
\newcommand{\bra}{\ensuremath{\langle}}
\newcommand{\ket}{\ensuremath{\rangle}}
%\def \ham {{\bf H}}
\def \ham {{\hat{H}}}
%\def \Sz {{\hat{\textrm{S}_{\textrm{z}}}}}
\def \Sz {{\hat{S}_z}}
\newcommand{\rff}[1]{{Eq.~\eqref{#1}}}
\def \Pgen {{P_{\textrm{gen}}}}
\def \Carb {{\textrm{C}_{\textrm{2}}}}
\def \Hij {{H_{\bvec{i}\bvec{j}}}}
\def \Kii {{K_{\bvec{i}\bvec{i}}}}
\def \Kij {{K_{\bvec{i}\bvec{j}}}}

\begin{document}
\title{Spectral functions of extended systems via quantum embedding}
\author{George~H.~Booth}
%\email{ghb24@cam.ac.uk}
\author{Garnet~Kin-Lic~Chan}  
\affiliation{Department of Chemistry, Frick Laboratory, Princeton University, Princeton, New Jersey 08544, USA}

\begin{abstract}
In a previous publication [PRL {\bf 109} 186404 (2012)], the ground-state density matrix embedding theory (DMET) was introduced, where in analogy with the dynamical mean-field method a 
set of local sites are self-consistently correlated with a bath describing the complete mean-field coupling to the rest of the extended system. Here, we extend the approach to introduce 
a frequency dependence, and demonstrate accurate spectral functions at a small 
computational cost. In a similar spirit to the frequency independent DMET method, the impurities are coupled via a small set of algebraically constructed, but now 
frequency-dependent bath states. The resultant spectral functions are directly obtained on the real-frequency axis, and allow for straightforward generalization both to 
multiple impurities and arbitrary perturbation operators. This vastly extends the scope and applicability of the DMET method in condensed matter problems as a 
cheap route to correlated local spectral functions of extended systems.
\end{abstract}
\date{\today}
\maketitle

Dynamic correlation functions are directly probed in spectroscopic techniques, and correspond to many of the important transport and wider electronic structure 
properties of materials. As such, their accurate theoretical construction is key for materials scientists, but few robust approaches exist for strongly correlated problems.
The difficulty is in requiring requiring both an accurate treatment of the electron correlations beyond mean-field band theory for the excited state components at each frequency, 
as well as the coupling of these states to the wider extended system. In dynamical mean-field theory (DMFT), the central quantity is the greens function of the system, which is
self-consistenty optimized 

Size of bath independent of the size of the underlying lattice. Only a function of the impurity cluster size and the perturbation, while still exact in the non-interacting limits.

%\bibliography{SpectralDMETBib}

\end{document}

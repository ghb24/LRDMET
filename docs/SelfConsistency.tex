\documentclass[a4paper,oneside,11pt]{article}
\usepackage{amssymb}
\usepackage{overcite}
\usepackage{amsmath}
\usepackage{enumerate}
\usepackage{graphicx}
\usepackage[nottoc]{tocbibind}
\usepackage{setspace}
\usepackage{rotating}
%\doublespacing
\parindent=0cm
%\parskip=.5cm
\numberwithin{equation}{section}

\newcommand{\rff}[1]{{Eq.~\ref{#1}}}

\newcommand{\half}{\ensuremath{\frac{1}{2}}}
\newcommand{\Rop}{\ensuremath{\mathbf{R}}}
\newcommand{\wfn}[2]{\ensuremath{\psi_{#1}{(#2)}}}
\newcommand{\VEC}[1]{\ensuremath{\mathbf{#1}}}
\newcommand{\pvec}{\ensuremath{\mathbf{p}}}
\newcommand{\rvec}{\ensuremath{\mathbf{r}}}
\newcommand{\kvec}{\ensuremath{\mathbf{k}}}
\newcommand{\Lvec}{\ensuremath{\mathbf{L}}}
\newcommand{\omegavec}{\mathbf{\omega}}
\newcommand{\drdr}{d\mathbf{r}_1d\mathbf{r}_2}
\newcommand{\latvec}[1]{\ensuremath{\mathbf{a}_{#1}}}
\newcommand{\Hamil}{\hat{H}}
\newcommand{\hamil}{\hat{h}}
\newcommand{\Tr}{\mathrm{Tr}}
\newcommand{\DMETBra}{\langle \mathrm{core}|\langle 0|}
\newcommand{\DMETKet}{|0\rangle| \mathrm{core} \rangle}
\newcommand{\Det}[1]{\ensuremath{D_{\mathbf{#1}}}}
\newcommand{\w}[1]{\ensuremath{w_{\mathbf{#1}}}}
\newcommand{\Etilde}[1]{\ensuremath{\tilde{E}_{\mathbf{#1}}}}
\newcommand{\vxc}{\ensuremath{\hat{v}_{\text{xc}}}}
\newcommand{\vext}{\ensuremath{\hat{v}_{\text{ext}}}}
\newcommand{\vhar}{\ensuremath{\hat{v}_{\text{har}}}}
\newcommand{\eigv}[1]{\ensuremath{\varepsilon_{#1}}}
\newcommand{\htdop}{\ensuremath{e^{-\beta\Hamil/P}}}
\newcommand{\rhoel}[1]{\ensuremath{\rho_{\mathbf{#1}}}}
\newcommand{\cyc}[1]{\ensuremath{(\mathbf{#1})}}
\def\Braket#1{\mathinner{\left\langle{#1}\right\rangle}}
\def\braket#1{\mathinner{\langle{#1}\rangle}}
\renewcommand{\thefootnote}{\fnsymbol{footnote}}

\def \bfX {{\bf X}}
\def \bfr {{\bf r}}
\def \bfR {{\bf R}}
\def \bfi {{\bf i}}
\def \ket {{\rangle}}
\def \bra {{\langle}}
\def \Gimp {{G_{imp} }}

\begin{document}

%\title{Self consistency in TD-DMET}
%\date{}
%\maketitle

\section*{\Large Self consistency in TD-DMET}

\subsection{Self consistency in DMFT}

There are certainly differences between the self-consistency in DMET and DMFT, but it will be worth recaping the DMFT formulation in the(my?!) adopted DMET language so that analogies can be drawn.
In DMFT, I {\em believe} the flow is as follows:
\begin{enumerate}
\item Calculate `non-interacting' greens-functions $g = [(\omega + i\delta)I - h - \Sigma]^{-1}$ projected over the impurity sites. The quotes are included since it is not really non-interacting, and potentially exact since the self-energy is included (in the atomic limit), but the nomenclature will remain. The sigma is repeated in each unit cell.
\item Construct a hybridization function, which will define the bath states, as $\Delta = (\omega + i\delta)I - h_{imp} - \Sigma - g^{-1}$, where $h_{imp}$ is the bare one-electron hamiltonian over the impurity space.
\item Fit the hybridization over $\omega$ to a set of one-electron bath states to define both the coupling to these states from the impurity, and the diagonal hamiltonian block of these states.
\item Solve impurity + bath system to obtain $\Gimp$ (no explicit self-energy contribution).
\item From $\Gimp$, form the local self-energy function, $\Sigma$, from $\Sigma = (\omega + i\delta)I - h_{imp} - \Delta - \Gimp^{-1}$.
\end{enumerate}

\subsection{Self consistency in DMET}

I will try to maintain the same structure of DMET self-consistency, although there are obviously key differences. Firstly, the `bath' in DMET will refer to the part of the many-body basis
constructed from the non-interacting part of the perturbation acting in the environment space. It will not refer to the one-electron bath orbitals formed in the 
ground-state DMET, which will remain unchanged. Obviously, since this bath is now a many-electron space, it is fundamentally different to the bath in DMFT. In addition, it is to be updated at
each frequency point and on the real-axis, rather than fitted across frequency on the imaginary axis. Despite these differences, a reasonably similar algorithm might proceed as follows:

\begin{enumerate}
\item Calculate `non-interacting' greens-functions as $g = [(\omega + i\delta)I - h - \Sigma]^{-1}$. We assume everywhere that the one-electron hamiltonian also includes the correlation potential from the ground state optimization. As in DMFT, the $\Sigma$ is copied through the space, and the resultant function projected over the impurity sites.
\item Construct the bath, parameterized by the function $[(\omega + i\delta)I - h - \Sigma - \Delta]^{-1}$ and projecting this into the core and virtual space of the Schmidt basis of the ground state DMET. Unlike previously, the bath orbitals are also parametrized by the action of the hybridization potential, to be defined later (though we could just use the non-interacting greens-function as before).
\item Solve the embedded + bath system to obtain $\Gimp$. Unlike DMFT, there are some choices to be made here about the hamiltonian. The hamiltonian should include the hybridization function in the one-electron
space that couples the embedded space to the bath functions, and in the diagonal bath blocks of the hamiltonian. The self-energy should not be included. Note that this will change the core energy of the bath states relative to the embedded space, which was previously overlooked - this should be key in coupling the states more effectively.
\item Calculate a new self-energy as $\Sigma = (\omega + i\delta)I - h_{imp} - \Delta - \Gimp^{-1}$. 
\item Calculate a hybridization function as $\Delta = (\omega + i\delta)I - h_{imp} - \Sigma - g^{-1}$
\item Repeat. Self-consistency achieved when $\Gimp$ is no longer changing?
\end{enumerate}

There are some choices here, such as whether the hybridization is used in the construction of the bath states, rather than just defining the coupling, but these small points should be able to be worked out. Nicely, this also only requires one lattice diagonalization per iteration. Does this sound at all reasonable?

\end{document}
